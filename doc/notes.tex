%   Dylan Wright - dylan.wright@uky.edu
%   EE480 - Assignment 4: Floating
%   note.tex : Implementer's Note

\documentclass[conference]{IEEEtran}
\usepackage{graphicx}
\usepackage{hyperref}

\begin{document}
\title{Assignment 4: Floating\\Implementor's Notes}
\author{\IEEEauthorblockN{Dylan Wright}
        \IEEEauthorblockA{dylan.wright@uky.edu}
        \IEEEauthorblockN{Robert McGillivary}
        \IEEEauthorblockA{}
        \IEEEauthorblockN{Evan Whitmer}
        \IEEEauthorblockA{}}

\maketitle

\begin{abstract}
This document serves as the Implementor's Notes for Team2C's EE 480 Assignment
4. This document intends to describe the methodology used to implement the 
project as well as any difficulties and important details regarding this task.
\end{abstract}

\section{Definitions and Standards}
\subsection{IDIOT}
IDIOT (Instruction Definition In Our Target) is the instruction set implemented
by this project. IDIOT is the instruction set created for the Spring 2016 
version of EE 480.\cite{idiot-isa}
\subsection{AIK}
AIK (Assembler Interpretor from Kentucky) is the tool used to create assembly
language programs in the IDIOT ISA.\cite{aik}
\subsection{BAD}
BAD (Barely Acceptable Decimal) is the floating point format implemented by
this project. BAD is a specification developed for the Spring 2016 version
of EE 480.\cite{idiot-float} This format is a variant of IEEE Std 754 
\texttt{binary32}.\cite{ieee-754} BAD drops all subnormal values except for
positive zero and truncates the 16 least significant bits of the significand.

\section{Implementation}
\subsection{Framework}
This is a Make project. The entire project can be built using the root Makefile
and any directory with a Makefile can be built using that. The project is split
into a \texttt{doc/}, 
\texttt{IDIOT/} and \texttt{verilog/} directory.
\subsubsection{\texttt{doc/}}
This directory is dedicated to documentation. In particular it contains the
\LaTeX source files for the implementor's notes (this document).
\subsubsection{\texttt{IDIOT/}}
This directory is dedicated to the IDIOT specification and the test program
build framework. Of particular interest are the following files:
\paragraph{\texttt{aik.py}} This file is a python3 script that can be used
to interface with the AIK cgi interface hosted on \texttt{aggregate.org}.
This script greatly simplifies the process of building individual test 
programs.
\paragraph{\texttt{build.sh}} This file is a bash script that can be used
to build all .idiot programs in the directory. This produces the .vmem files
which can be used to exercise the processor.
\subsubsection{\texttt{verilog/}}
This directory is dedicated to the verilog implementation of the project and
the framework used to test it. Of particular interest are the following files:
\paragraph{\texttt{pipe.v}} This file is the verilog source code which 
implements the processor. 
\paragraph{\texttt{test.sh}} This file is used to test the processor. At
the moment it runs the unit tests in \texttt{tests/testbenches/}.

\subsection{Floating Point Instructions}
The following instructions were added to the previously implemented integer
instructions. They are described in the order they were implemented.
\subsubsection{\texttt{i2f}}
(integer to float) is an instruction which converts a 16-bit
integer to a 16-bit float. This instruction converts the x signal of the ALU.
This is done by normalizing the integer, determining
the exponent, sign. For the subnormal value positive 0, the input is passed
directly through the ALU.
\subsubsection{\texttt{f2i}}
(float to integer) is an instruction which converts a 16-bit
float to a 16-bit integer. This instruction converts the x signal of the ALU.
This is done by denormalizing the significand and adjusting the signal if the
sign bit is set. 
%For the subnormal value positive 0, the input is passed directly through the ALU.
\subsubsection{\texttt{mulf}}
(multiply float) is an instruction which multiplies two 16-bit
floats. This instruction multiplies the x and y signals of the ALU. This is 
done by adding the exponents, multiplying the significands, and setting the
sign bit. For the subnormal value positive 0, the output is set to positive
0.
\subsubsection{\texttt{addf}}
(add float) is an instruction which adds two 16-bit floats. This
instruction adds the x and y signals of the ALU. This is done by denormalizing
the float with the smaller exponent, adding or subtracting the significands,
and normalizing the result. For the subnormal value positive 0, the input which
is not positive 0 is passed directly through the ALU.
\subsubsection{\texttt{invf}}
(invert float) is an instruction which computes the inverse of a 16-bit
float. This instruction inverts the x signal of the ALU. This is done using
a look up table provided for the assignment.\cite{invf-lut} For the special
case of the subnormal positive zero, zero is passed through the alu.
\subsection{Pipeline Integration}
The floating point alu was built into the alu module from the previous project.
This module is instantiated by the pipelined processor provided as the 
solution to the previous project.\cite{a3-sol}

\section{Issues}
In order to use the provided \texttt{idioc} compiler, the following changes
were made:
\begin{itemize}
    \item Fixed issue with function labels missing an underscore
    \item Fixed issue with data words being output as a list of values which
        was not being accepted by AIK
\end{itemize}

\section{Testing}
\subsection{Continuous Integration}
In an effort to increase productivity and test usefulness, this group has
been using a Jenkins continuous integration server. This server is running
on a server Dylan Wright operates.\footnote{\url{intellproject.com:8090}} 
The server is configured to attempt to
build whenever the project GitHub receives a commit. The build process makes
all source files and reports if any step fails. Additionally it runs the
unit test script (\texttt{verilog/test.sh}) which will cause the build to fail
if any test case fails. The testscript also outputs a junit xml report to
\texttt{reports/junit.xml}. The xml file is used to generated test result
graphs by Jenkins. 

\subsection{Floating Point Unit Testing}
The test script tests floating point operations with the unittests in the
\texttt{verilog/tests/testbench/} directory. The files with the .v extensions
are inserted into the testbench in the main verilog file. 
\subsection{Pipeline Program Testing}
The test script tests the pipeline by using the unittests in the 
\texttt{verilog/tests/testprogs/} directory. The files with the .vmem 
extensions are concatenated together and the resulting file is used to 
initialize the memory. The testbench in the main verilog file by default (not
floating point unittests) resets the processor and runs until the processor
halts. 

\section{Philosophy}
\subsection{Sufficiency}
The floating point instruction set is sufficient for the context of this
project. Implementing features like comparison in hardware would be a nice
addition but can be easily done by the compiler instead of the hardware. By
leaving much of the complexity in the software domain the hardware is simpler
to design and test.

\subsection{Necessity}
The floating point instructions implemented here represent a distinct advantage
over implementation in software. These instructions occur often enough to
warrant implementation in hardware. Taking more than one cycle to perform
a floating point multiply or addition would significantly affect the 
performance of certain programs.

\subsection{Usefulness}
The format implemented here is mostly academic but it is used in some real 
world situations. The larger dynamic range of this format is particularly 
advantageous. While binary 16 is also a useful format, the top half of binary 32
is equally useful in this context.

\bibliographystyle{IEEEtran}
\bibliography{notes}
\end{document}
